% !TEX TS-program = pdflatex
% !TEX encoding = UTF-8 Unicodeo
%\documentclass[3p,times,table]{elsarticle}
\documentclass[3p,times,table]{article}
\usepackage[hmarginratio=2:3,top=32mm,left=20mm,columnsep=17pt]{geometry}
%=============================================================================
%==========  A L L    I N C L U D E S
%%%%%%%%%% PACKAGES %%%%%%%%%% 
%\usepackage[utf8x]{inputenc}
%\usepackage{epstopdf} %converting to PDF
%\usepackage{array}
\usepackage{titling}
\usepackage[svgnames]{xcolor} 
\usepackage{tabularx}
\usepackage{graphicx} 
\usepackage{amsmath}
\usepackage{amssymb}
\usepackage{amsfonts}	
\usepackage{moreverb}
\usepackage{dsfont}
\usepackage{tipa}
\usepackage{upgreek}
%\usepackage{grffile}
\usepackage{bm}
\usepackage{multirow}
\usepackage{soul}
\usepackage{ textcomp }
\usepackage{relsize} % e.g. used for \mathsmaller
\usepackage{caption}
\usepackage{subcaption}

\renewcommand{\tabularxcolumn}[1]{m{#1}}
\newcommand\BibTeX{{\rmfamily B\kern-.05em \textsc{i\kern-.025em b}\kern-.08em
T\kern-.1667em\lower.7ex\hbox{E}\kern-.125emX}}

% NATBIB options:

\usepackage[numbers]{natbib}

% END NATBIB options.

% BIBLATEX options:
%\usepackage[backend=bibtex,giveninits=true,sorting=nyt,url=false,doi=true,eprint=false,isbn=false,
%backref,backrefstyle=none,maxbibnames=99]{biblatex}
%\DefineBibliographyStrings{english}{%
%  backrefpage = {Cited on p\adddot},%
%  backrefpages = {Cited on pp\adddot}%
%}
%
%\bibliography{TwoPhaseSolidFluid}
%% \bibliography{/home/utilisateur/Dropbox/library}
%%\bibliography{/home_pers/peshkov/Dropbox/library}
%
%\renewcommand*{\bibfont}{\footnotesize}
%
%% in order to suppress 'In:'
%\renewbibmacro{in:}{%
%  \ifboolexpr{%
%     test {\ifentrytype{article}}%
%  }{}{\printtext{\bibstring{in}\intitlepunct}}%
%}
% END BIBLATEX options.


%%%%%%%%%% PACKAGES %%%%%%%%%%

\renewcommand{\tabularxcolumn}[1]{m{#1}}

% To have colored cited papers, hyperlinked to the 
% bibiography, help to know if papers are not cited
% but in the bibliography still
\usepackage{hyperref} 
\hypersetup{
    colorlinks=true,                          
    linkcolor=blue, % Couleur des liens internes
    citecolor=DarkRed, % Couleur des num�ros de la biblio dans le corps
    urlcolor=blue  } % Couleur des url
%\usepackage[hyperpageref]{backref} 
%usepackage[square,numbers]{natbib}
%\RequirePackage[hyperpageref]{backref}
%\backreffrench
%\renewcommand*{\backref}[1]{}  % Disable standard
%\renewcommand*{\backrefalt}[4]{% Detailed backref
 %\ifcase #1 %
 %\relax%(Not cited.)%
  %\or
%% (Cit\'e page~#2.)%
 %(Cited page~#2.)%
 %\else
 %%(Cit\'e pages~#2.) 
 %(Cited page~#2.)%
 %\fi}

\setlength{\oddsidemargin}{.5cm} \setlength{\evensidemargin}{.5cm}
\setlength{\textwidth}{15cm} \setlength{\textheight}{21.0cm}
\setlength{\topmargin}{0in}

%%%%%%%% NEW COMMANDS %%%%%%%%
\newcommand{\pd}{\partial}
\newcommand{\ce}{{\varepsilon}}
\newcommand{\xx}{{\boldsymbol{x}}}
\newcommand{\yy}{{\boldsymbol{y}}}
\newcommand{\zz}{{\boldsymbol{z}}}
\newcommand{\x}{\mbf{x}}
\newcommand{\y}{\mbf{y}}
\newcommand{\Th}{\mathcal{T}_h}				%mesh notation
\newcommand{\nij}{\mathbf{n}_{ij}}				%outward unit normal to S_ij
\newcommand{\Eel}{\mathcal{E}_{el} }			%indices set of the real cells
\newcommand{\Ebd}{\mathcal{E}_{bd} }			%indices set of the virtual cells
\newcommand{\tEel}{\widetilde{\mathcal{E}_{el}}}	%indices set of the real and virtual cells
\newcommand{\Epc}{\mathcal{E}_{pc} }			%indices set of problematic cells
\newcommand{\Unui}{\underline{\nu}(i)}		%indices set of cells linked to K_i by a side
\newcommand{\Onui}{\overline{\nu}(i)}			%indices set of every cells linked to K_i
\newcommand{\urec}{\widetilde u}				%polynomial rec sur K
\newcommand{\Urec}{\widetilde U}				%polynomial rec sur K
\newcommand{\R}[2]{{\mc{R}}^{\{#1,#2\}}} %NEEDED FOR POLY REC ..
\newcommand{\CPD}{\textsf{CellPD}}
%\newcommand{\un}{\textrm{1}}
\newcommand{\un}{1 \hskip -3pt \textrm{I}}
\newcommand{\Deg}[1]{ \mathsf{d}_{#1} }
\newcommand{\EPD}{\textsf{EdgePD}}
\newcommand{\FPD}{\textsf{FacePD}}
\newcommand{\EPDMeth}[1]{$\mathsf{EPD}_{\mathsf{#1}}$}
\newcommand{\mc}[1]{\mathcal{#1}}			% Simplification of usefull calligraphies
\newcommand{\mbb}[1]{\mathbb{#1}}			%
\newcommand{\mbf}[1]{\mathbf{#1}}			%
\newcommand{\msf}[1]{\mathsf{#1}}		       %
\newcommand{\mait}[1]{\mathit{#1}}			%
\newcommand{\mfrk}[1]{\mathfrak{#1}}		%
\newcommand{\tbf}[1]{\textbf{#1}}				%
\newcommand{\tsf}[1]{\textsf{#1}}				%
\newcommand{\tit}[1]{\textit{#1}}					%
\newcommand{\trm}[1]{\textrm{#1}}					%
\newcommand{\noi}{\noindent}
\newcommand{\Frac}{\displaystyle\frac}
\newcommand{\Int}{\displaystyle\int}
\newcommand{\Sum}{\displaystyle\sum}
\newcommand{\Bigcup}{\displaystyle\bigcup}
\newcommand{\Max}{\displaystyle\max}
\newcommand{\Min}{\displaystyle\min}
\newcommand{\Eq}[1]{equation {(\ref{#1})}}

\newcommand{\Ell}{\mathcal{L}}
\newcommand{\emm}{m}
\newcommand{\dxx}[1]{ \partial_{xx} #1 }
\newcommand{\dyy}[1]{ \partial_{yy} #1 }
\newcommand{\dzz}[1]{ \partial_{zz} #1 }
\newcommand{\HRule}[1]{ {\centering \rule{#1\linewidth}{0.2mm}} }
\newcommand{\DMPutwo}{$[\text{DMP}\!\to\!u2]$}
\newcommand{\PADDMPutwo}{$[\text{PAD}\!\to\!\text{DMP}\!\to\!u2]$}
\newcommand{\red}[1]{{\color{red} #1}}
\newcommand{\blue}[1]{{\color{blue} #1}}
\newcommand{\mygreen}{\textcolor[rgb]{0.0,0.60,0.}}
\newcommand{\myorange}{\textcolor[rgb]{0.6,0.0,0.}}
\newcommand{\oz}[1]{ \textcolor{red}   {\texttt{\textbf{OZ: #1}}} }
\newcommand{\rmd}{{\rm d}}



\newcommand{\bdm}{\begin{displaymath}}
\newcommand{\edm}{\end{displaymath}}

\newcommand{\bea}{\begin{eqnarray} }
\newcommand{\eea}{\end{eqnarray} }

\newcommand{\apriori}{\textit{a priori} }
\newcommand{\aposteriori}{\textit{a posteriori} }

\newcommand{\dev}{\textnormal{dev}} 

\renewcommand{\AA}{{\bm{A}}}
\renewcommand{\aa}{{\bm{a}}}
\renewcommand{\ggg}{{\bm{g}}}
\newcommand{\DD}{{\bm{D}}}
\newcommand{\HH}{{\bm{H}}}
\newcommand{\sfA}{{\mathsf{A}}}
\newcommand{\sfB}{{\mathsf{B}}}
\newcommand{\GG}{{\boldsymbol{G}}}
\newcommand{\rr}{{\bm{r}}}
\newcommand{\ee}{{\bm{e}}}
\newcommand{\bb}{{\bm{b}}}
\newcommand{\hh}{{\bm{h}}}
\newcommand{\dd}{{\bm{d}}}
\newcommand{\vv}{{\bm{v}}}
\newcommand{\uu}{{\bm{u}}}
\newcommand{\mcE}{{\mathcal{E}}}
\newcommand{\calI}{\mathcal{I}}
\newcommand{\EE}{{\bm{E}}}
\newcommand{\BB}{{\bm{B}}}

\newcommand{\FF}{{\bm{F}}}
\newcommand{\II}{{\bm{I}}}
\newcommand{\JJ}{{\bm{J}}}
\newcommand{\QQ}{{\bm{Q}}}
\renewcommand{\SS}{{\bm{S}}}
\newcommand{\PP}{{\bm{P}}}
%\newcommand{\SS}{{\boldsymbol{S}}}
\newcommand{\WW}{{\bm{W}}}
\newcommand{\ww}{{\bm{w}}}
\newcommand{\wbf}{{\bm{w}}}
\newcommand{\pp}{{\bm{p}}}
\newcommand{\qq}{{\bm{q}}}

\newcommand{\Id}{{\bm{I}}}
\newcommand{\tr}{\textnormal{tr}}
\newcommand{\BS}{{\boldsymbol{\sigma}}}
\renewcommand{\Re}{\textnormal{Re}}
\newcommand{\transpose}{{\rm {\mathsmaller T}}}
\newcommand*\samethanks[1][\value{footnote}]{\footnotemark[#1]}
\newcommand{\tort}{{\mathcal{T}}}
\newcommand{\Km}{K_{\rm m}}
\newcommand{\Ks}{K_{\rm s}}
\newcommand{\Ku}{K_{\rm u}}
\newcommand{\Kf}{K_{\rm f}}
\newcommand{\lambdau}{\lambda_{\rm u}}
\newcommand{\muu}{\mu_{\rm u}}
\newcommand{\mus}{\mu_{\rm s}}
\newcommand{\rhof}{\rho_{\rm f}}
\newcommand{\rhos}{\rho_{\rm s}}
\newcommand{\pf}{p_{\rm f}}
\newcommand{\ps}{p_{\rm s}}
\newcommand{\alphaf}{\alpha_{\rm f}}
\newcommand{\alphas}{\alpha_{\rm s}}
\newcommand{\cf}{c_{\rm f}}
\newcommand{\cs}{c_{\rm s}}
\newcommand{\vf}{v_{\rm f}}
\newcommand{\vs}{v_{\rm s}}
\newcommand{\Cs}{C_{\rm s}}
\newcommand{\Csh}{C_{\rm sh}}
\newcommand{\Cf}{C_{\rm f}}
\newcommand{\csh}{c_{\rm sh}}
\newcommand{\cshf}{c_{{\rm sh},1}}
\newcommand{\cshs}{c_{{\rm sh},2}}

%%%%%%%%%%%%%%%%%%%%%%%%%%%%%%%%%%%%%%%%%%%%%%%%%%%%%
%%%%%%%%%%%%%%%%%%%%%%%%%%%%%%%%%%%%%%%%%%%%%%%%%%%%%
%%%%%%%%%%%%%%%%%%%%%%%%%%%%%%%%%%%%%%%%%%%%%%%%%%%%%
% DOC BEGINNING

\newfont{\numerikEleven}{ecrm1000}
\newfont{\numerikTen}{cmss10}
\newfont{\numerikNine}{cmss9}
\newfont{\numerikEight}{cmss8}

% New commands for the revision :
%\usepackage[dvipsnames]{xcolor}
\newcommand{\revOne}[1]{\textcolor{Red}{#1}}
\newcommand{\revThree}[1]{\textcolor{Blue}{#1}}


%=========================================================================

\title{\bf Revision 1}
\date{}


\thanksmarkseries{arabic}
\begin{document} 
\maketitle


\subsection*{Answer to Reviewer 1}
The paper derives a thermodynamically consistent system of hyperbolic equations describing dynamic 
behaviour of a two-phase medium that consists of a fluid and solid constituents with the latter 
manifesting non-elastic response. The resulting model is a generalisation of the two-phase 
fluid-solid model described in subsection 2.2 of the present paper accounting for strength of the 
solid matrix. The characteristics parameters of the present system are compared with those of the 
Biot's system. The paper develops further the symmetric systems of hyperbolic conservative 
equations of Friedrichs type for a novel type of material and could be of interest to physicists 
and geophysicists analyzing seismic response of fluid-containing structures. The following comments 
seem to be relevant for the paper:
\\

\textbf{1.}      The mathematical formulation is stated quite clearly. However, the physical 
description of 
a medium considered is not easily readable. The title of the paper claims consideration of 
"multiphase mixtures", therefore the reader is trying to identify multiple phases with the 
understanding later on that only two phases are actually considered. As a suggestion, either the 
title claims description of the corresponding two-phase mixtures or a graph outlining a schematic 
of the material and its fluid and condensed phases with elasto-plastic response, is drawn up, which 
could facilitate the reading. In this case, when formulating the background models in subsections 
2.1 and 2.2, the authors could refer to such a schematic for specification of the phases described 
with the corresponding background models, which could make the reading and understanding the paper 
objective easier. Another misunderstanding is introduced by title of subsection 2.1, which makes 
the reader to look for
solid and fluid phases in the background model. In reality, the model described in the subsection 
deals with a single-phase medium. It would be easier to read if the authors adjust the title and 
later, in the body of the subsection, inform the reader that the model is applicable to both fluid 
and solid media, which is controlled by the shear stress relaxation 
time.
\\
\revOne{Thank you. We have changed the title of the paper, referring 
the theory to a two-phase solid- liquid mixture.
	We also changed the name of section 2.1 and added there a 
	sentence 
	explaining the meaning of the unified continuum model (in red).
}
\\

\textbf{2.}      Some small typos and denotation deficiencies need to be addressed. For example, 
the kinetic 
energy denotation after equation (10) on page 5 should be $ E_3 $ (not $ E_2 $), $ a_{mn} $ in 
equation 
(27) is 
not introduced (possibly, it relates to $ A_{mn} $, but it is not clear), the derivative of energy 
with 
respect to mass concentration at the end of first paragraph on page 5 refers to a wrong phase, $ 
s_0 $ 
in equation (28b) is not defined (it seems that subscript '0' in this denotation is redundant 
because the entropy perturbation used later in (33g) is s). The tilded $ \sigma $ in equation (22) 
is 
not introduced: the reader might guess that this is stress in the solid matrix, but this is not 
confirmed. This denotation is used later on throughout the paper, for example, in equation (31). 
This symbol is redesignated by $ s_{ij} $ from (37) and on, which results in extra confusion. I 
would 
suggest the authors to inspect the denotations once more in order to 
avoid confusions.
\\
\revOne{Thank you. In the revised manuscript, $ E_2 $ is replaced by 
$ E_3 $.
	The definition of  $ a_{mn} $ was missed. Now it is presented 
	after equation (25).
	In the end of Section 2, the derivative of energy with respect 
	to 
	the first phase mass fraction is corrected.
	$ s_0 $ in (29) is replaced by $ s $.
	We introduce the single notation $ s_{ij} $ everywhere, instead 
	of 
	$ \tilde{\sigma}_{ij}$  and $ s_{ij} $ in (38)
}
\\

\textbf{3.}      A comment on temperature limitations for the present formulation would be useful. 
It is 
unlikely that the definition of single entropy employed in the present model can be used at extreme 
temperatures because temperatures of the phases may be different even at the same entropy due to 
different compression capacities of the phases. Possible definition of phase temperatures (which 
are not needed for the formulation but could be required by an application) results in the 
averaging for the mixture temperature without the weighting involving the heat capacity 
coefficients, which may be required in the case of equilibrating the temperatures. Therefore, a 
comment on applicability of the model to the quasi-isentropic processes could ensure validity of 
the formulation for the present applications.
\\
\revOne{At the beginning of Section\,2.2, we added a following 
comment 
about temperature limitation for the two-phase compressible flow. 
	`\textit{In our consideration, we restrict ourselves to a single entropy approximation for small variations of phase temperatures. In Ref.[39], it is proved that 
for multiphase compressible mixtures a single entropy approximation is suitable for the flows close to thermal equilibrium. In this case, the change of phase temperatures due to the small variation of phase entropies is negligibly small.
Finally note, that we consider only small density variations which means that the processes under consideration are close to thermal equilibrium.
	}''\\
	Also, a phrase``\textit{As noted at the beginning of Section 
	2.2, 
	for 
	processes which are close to thermal equilibrium, the 
	single-entropy approximation is acceptable.}'' is added in 
	Section 
	2.3 in the description of two-phase solid/fluid model.
}
\\

\textbf{4.}      The shear deformation energy in equation (18) is obviously related to the matrix 
phase 
only. The footnote on page 7 refers the corresponding speed of shear waves for the mixture to 
$ c_{sh} $. On the other side, the authors define the distortion tensor A (and associated tensor g 
in 
(18)) only for the mixture. Therefore, there is some unclarity with the thermodynamical calculation 
of the shear stresses and corresponding speed for the solid phase. Specifically, is $ c_{sh} $ 
introduced 
in subsection 2.1 and used in (18) actually $ c_{sh} $ mentioned in the footnote? Some comment or 
brief 
explanation would help in understanding of this part of the paper.
\\
\revOne{We added  to Sect. 2.3 (after eq.(18))  a justification of 
the definition of 
shear energy (in red).}
\\

\textbf{5.}      I would recommend an enlargement of figure legends for figures 3-6, 8, 10, and 11 
in order 
to make them more readable.
\\
\revOne{Thank you. The legends for figures 3-6, 8, 10-12 have been 
expanded}.
\\

\textbf{6.}      Specification of the test problems in the last paragraph of conclusion is 
required. A 
simple limitation of the problems to those within the approximation of small-amplitude wave 
propagation would be sufficient.
\\
\revOne{Thank you. A specification of the test problems added in 
Conclusions Section.}
\\

The paper contains an interesting and novel formulation of a two-phase fluid-solid model and could 
be useful to scientists in the relevant area of science. After clarification of the comments above, 
requiring a minor revision, the paper can be recommended for publication without further revision.
\newpage

\subsection*{Answer to Reviewer 3}
The purpose of the manuscript seems to be to formulate evolution of porous media within the 
Symmetric Hyperbolic Thermodynamically Compatible (SHTC) equations and to compare the model with 
the standard Biot equations. Both goals have clearly been achieved and superiority of the SHTC 
model is clearly presented. I believe that further discussion of the modeling part could improve 
readability of the manuscript, which is why I suggest a minor revision, see comments below.
\\

\textbf{1.} The SHTC equations are usually derived by a 
Lagrange-Euler transformation. However, 
another 
derivation of at least some parts of SHTC is possible, based on kinetic theory (in the sense of 
Liouville equation). For instance in monography [1] or paper [2] a theory of mixtures is derived 
from the Liouville equation, which is equipped with two momenta (each for one species). This seems 
to be in close relation with Eqs. 21 (dropping the distortion, however), but there is a difference. 
In the approch coming from Liouville equation both momenta are advected by themselves (i.e. the 
right hand side of the evolution equation for the first momentum contains derivative of energy with 
respect to the the momentum and analogically for the second momentum). These self-advection terms 
are reversible, they do not raise entropy. Similar terms also appear in theories of mixtures based 
on extended irreversible thermodynamics, see e.g. [3]. On the other hand, the equation for w, e.g. 
(9e),
contains no such term. The term, however, would be invisible in 1D simulations, as it would be 
proportional to the curl of w (as in the heat conduction model in [1]). Could you please comment on 
this difference?
\\
\revThree{Thank you. Indeed, such a term is missing in the equation 
for $ \ww $. Nevertheless, it should not affect the resulting 
linearized system and should vanish. We believe the differences in 
solution with and without such a term should appear when considering 
finite strain deformations. That would be interesting to consider 
such a possibility in the future. We added a comment discussing this 
aspect in the conclusion.}

\textbf{2.} The equations themselves seem to be more or less standard in the context of SHTC 
literature. 
Comparison with Biot equations were made for instance in [4]. Could you please explicitly declare 
novelty of the manuscript?
\\
\revThree
{Thank you. We now explicitly listed the novelties of the 
current 
manuscript in the Introduction (in blue).}
\\

Minor details:\\

\textbf{3.} Why is in Eq. 9 the w-field with an upper index? From 
the construction of the SHTC 
equation it 
should be equipped with a lower index, as in the terms with 
derivative of energy with respect to w.
\\
\revThree{The positioning of the indexes for vectors is not related 
to the tensor notations but we merely reserve the subscripts in the 
velocities for denoting the phase, 1 or 2 to make it consistent with 
subscripts in $ c_1 $, $ \alpha_1 $, while the superscript in 
velocities is 
reserved for denoting the vector components. A comment was added 
after system (1) in blue.}
\\

\textbf{4.} In Eqs. 21 there are three velocity-like fields, $ \ww 
$, 
$ \vv_1 $ and $ \vv_2 $ although 
there were 
only two ($ \vv $ 
and $ \ww $) in the preceding equations. Could you please clarify 
what are 
the state variables in Eqs. 21?
\\
\revThree{Indeed, the velocities $ \vv_{1} $ and $ \vv_2 $ are the 
velocities of the constituents, their components denoted as $ v_1^i 
$ and $ v_2^i $, $ i=1,2,3 $ correspondingly. They were introduced 
in Sec.2.3. We write them now more explicitly as an outline formula, 
see 
eq.(15).}

References:
\\

[1] Pavelka, Klika, Grmela, Multiscale Thermo-Dynamics, de Gruyter 2018.\\

[2] Pavelka, M., Klika, V., Esen, O. and Grmela, M., A hierarchy of Poisson brackets in 
non-equilibrium thermodynamics, Physica D: Nonlinear Phenomena 335 (2016), pages 54-69 \\

[3] Pavelka, M., Marsik, F., Klika, V., Consistent theory of mixtures on different levels of 
description, Int. J. Eng. Sci. 78C (2014), pp. 192-217 \\

[4] Elena Vazquez-Cendon, Arturo Hidalgo, Pilar Garcia-Navarro \& Luis Cea: Numerical Methods for 
Hyperbolic Equations:Theory and Applications, CRC press 2013

%\printbibliography
%\bibliographystyle{plainurl}
%\bibliography{library}







\end{document}
